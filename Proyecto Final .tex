\documentclass[conference]{IEEEtran}
\usepackage[utf8]{inputenc}
\usepackage{textcomp}
\usepackage[english]{babel}
\usepackage{amsmath}
\usepackage{amsfonts}
\usepackage{amssymb}
\usepackage{graphicx}
%\usepackage{xcolor}
% Dependencies for Excel2Latex
\usepackage[table]{xcolor}
\usepackage{booktabs}
\usepackage{multirow}
% Dependencies for Excel2Latex
\usepackage{listings}
%Dependencies for [1.]
\usepackage{enumerate}
%Dependencies for [1.]
\usepackage{tikz}
\usepackage{float}
\usepackage{karnaugh-map}
\usepackage{adjustbox}
\usepackage[left=1cm,right=1cm,top=1cm,bottom=1cm]{geometry}
%Habilita bookmarks en PDF
\newcommand\MYhyperrefoptions{bookmarks=true,bookmarksnumbered=true,
pdfpagemode={UseOutlines},plainpages=false,pdfpagelabels=true,
colorlinks=true,linkcolor={black},citecolor={black},
urlcolor={black}}
\usepackage[\MYhyperrefoptions]{hyperref}
%Titulo del documento
\title{Proyecto final: Aplicación para pizzeria}

\makeatletter
\newcommand{\linebreakand}{%
  \end{@IEEEauthorhalign}
  \hfill\mbox{}\par
  \mbox{}\hfill\begin{@IEEEauthorhalign} \hfill
}
\makeatother

\renewcommand\thesection{\arabic{section}}
\renewcommand\thesubsection{\thesection.\arabic{subsection}}
\renewcommand\thesubsubsection{\thesubsection.\arabic{subsubsection}}

\author{
	\IEEEauthorblockN{\hfill Kenneth Abarca Coronado \hfill}
	\IEEEauthorblockA{\textit{Estudiante Ing. en Sistemas de Computación}\\ 
	\textit{Universidad Fidélitas}\\
	San José, Costa Rica \\
	\href{mailto:Kabarca20607@ufide.ac.cr}{Kabarca20607@ufide.ac.cr}}
\and
	\IEEEauthorblockN{\hfill Jonathan Chavarria Peña \hfill}
	\IEEEauthorblockA{\textit{Estudiante Ing. en Sist. Computación}\\ 
	\textit{Universidad Fidélitas}\\
	San José, Costa Rica \\
	\href{mailto:jonach1998@gmail.com}{jonach1998@gmail.com}}
\and
	\IEEEauthorblockN{\hfill Erick Corrales Montero\hfill}
	\IEEEauthorblockA{\textit{Estudiante Ing. en Sist. Computación}\\
	\textit{Universidad Fidélitas}\\
	San José, Costa Rica \\
	\href{mailto:ecorrales00712@ufide.ac.cr}{ecorrales00712@ufide.ac.cr}}
\linebreakand % <------------- \and with a line-break
	\IEEEauthorblockN{\hfill Marco Fonseca Solorzano \hfill} 
	\IEEEauthorblockA{\textit{Estudiante Ing. en Sist. Computación}\\
	\textit{Universidad Fidélitas}\\
	San José, Costa Rica \\
	\href{mailto:marcosfin0232@gmail.com}{marcosfin0232@gmail.com}}
\and
	\IEEEauthorblockN{\hfill Keren Jimenez Fernandez \hfill} 
	\IEEEauthorblockA{\textit{Estudiante Ing. en Sist. Computación}\\
	\textit{Universidad Fidélitas}\\
	San José, Costa Rica \\
	\href{mailto:kjimenez80215@ufide.ac.cr}{kjimenez80215@ufide.ac.cr}}
\and
	\IEEEauthorblockN{\hfill Sebastián Lizano Fernández \hfill} 
	\IEEEauthorblockA{\textit{Estudiante Ing. en Sist. Computación}\\
	\textit{Universidad Fidélitas}\\
	San José, Costa Rica \\
	\href{mailto:slizano40347@ufide.ac.cr}{slizano40347@ufide.ac.cr}}
\linebreakand % <------------- \and with a line-break
	\IEEEauthorblockN{\hfill Valeria Morales Cordero\hfill}
	\IEEEauthorblockA{\textit{Estudiante Ing. en Sist. Computación}\\
	\textit{Universidad Fidélitas}\\
	San José, Costa Rica \\
	\href{mailto:valemc0603@gmail.com}{valemc0603@gmail.com}}


}


%Inicio del documento
\begin{document}

\maketitle

%Agrega numeracion a las paginas
%\thispagestyle{plain}
%\pagestyle{plain}

%\begin{abstract}
	
	
%\end{abstract}




\section{Antecedentes}

Hoy en día restaurantes famosos (Burger King, McDonald’s, TacoBell, KFC, entre otros) están mejorando mucho a tal punto de que usan aplicaciones ya sea para tener el control del inventario, productos, para un fin administrativo, ordenar comida, entre otros. También la situación actual del país causada por la pandemia del Covid-19 genera que mucha gente no vaya a los restaurantes por miedo a contagiarse y optan por usar aplicaciones para pedir comida desde la comodidad de su casa. La aplicación de la pizzería ayudará al restaurante mejorar sus servicios, mejorar el orden en su economía, expandir su zona de venta. Así como beneficiar a los clientes de pedir desde la comodidad de su casa y con todos los servicios que ofrece el restaurante.

\section{Marco Legal}

Para proteger el programa creado por la empresa, este se inscribirá como propiedad intelectual, de esta manera el uso del código quedara a discreción de la desarrolladora.

La venta o las licencias que se darán a terceros del programa se realizarán bajo los términos y condiciones de un contrato entre el cliente y la empresa, en el contrato se estipularan los siguientes puntos:

\begin{itemize}
\item Garantías
\item Actualizaciones
\item Servicio Técnico
\item Mejoras especializadas para el cliente
\item Capacitaciones necesarias para utilizar la app
\item Costo del programa
\item Costo del servidor y base de datos (de ser requerido)
\item Duración del contrato
\end{itemize}

Las licencias no serán exclusivas para el cliente, esto quiere decir que el programa se podrá vender a otras empresas, salvo alguna excepción estipulada en un contrato. El cliente que adquiera el software podrá utilizarlo única y exclusivamente para uso de su empresa no podrá dar licencias ni usarlo para programas de capacitación a otras empresas.

La desarrolladora entregará la documentación externa del software al cliente, con el fin del uso correcto del mismo, asimismo minimizar los problemas que se puedan generar al instalar o utilizar el programa.

\section{Ubicación del proyecto}

La pizzería llamada D´Pelicula a la que se le venderá el software esta ubicada en San José, Goicoechea, Ipis, Zetillal, a 300 metros Este de la comisaria de Zetillal.
La misma solo tiene una sucursal por el momento.

\section{Justificación del proyecto}

Con todo el avance que ha tenido las tecnologías esto ha provocado un incremento en el mercado ya que hay más competencia en todos los sectores. unos de los problemas que se encuentra en este proyecto es la falta de la tecnología, lo que podemos lograr o conseguir es entrar en el mercado con ayuda de una app ya que el 80 % de la población usa celular y la mayoría son smartphones. 
La idea principal es poder resolver el problema con una app, que abarque los dispositivos tanto IOS, ANDROID u otro sistema operativo que este en el mercado, para asegurar un mercado amplio lo que hacemos es una app para que cualquier persona con cualquier dispositivo entre y pueda ordenar fácil y rápido esto ayudara a mejorar las ventas del cliente y además a que se pueda manejar de una manera más ordenada. También poder dar una oportunidad al cliente de poder competir en el mercado y poder ampliar todos sus servicios, esto ayudara a tener una mejora en su atención al cliente y una mejora económica a su restaurante. 

\section{Objetivos}
\subsection{Objetivo General}

Desarrollar una aplicación de pedidos y facturación para una pizzería con la finalidad de mejorar los servicios que brinda la pizzería.

\subsection{Objetivos Específicos}
\begin{itemize}
\item Definir los servicios que el cliente desea tener en la aplicación.
\item Crear el software con las especificaciones antes establecidas.
\item Realizar la documentación externa necesaria para el cliente.
\end{itemize}


\section{Descripción del proyecto}

El proyecto se basa en la creación de una aplicación para una pizzería; el cliente podrá registrarse a la aplicación, podrá elegir para comer en el restaurante o que sea a domicilio; tendrá un menú de pizzas a elegir: pastas, refrescos entre otros, código promocional para un descuento e imprimir o guardar la factura de la compra. 
También con la aplicación tendrá un uso administrativo; los que tengan acceso como modo administrador podrá ver las compras, ventas, usuarios y contraseñas registrados, modificar el menú, cambiar precios, entre otros.
Se pretende crear un software que sea fácil de usar, que tenga un interfaz agradable para el usuario, incluso crear un usuario y contraseña tanto para el cliente como para el administrador.


\section{Fuente de financiamiento}

La desarrolladora de software solicitará un préstamo al banco para cubrir los salarios de los ingenieros que trabajan en la empresa, también para la adquisición de los servidores o bases de datos (de ser requerido); el mismo será cancelado con el pago que realicé el cliente cuando se le entregue el software final.



\section{Actividades de ejecución de la obra}

\begin{enumerate}

\item Establecer con el cliente las especificaciones del programa.
\item Crear el contrato con todos los términos y condiciones.
\item Iniciar con el financiamiento del proyecto.
\item Crear el software.
\item Realizar las pruebas al programa.
\item Crear la documentación interna y externa del programa.
\item Entregar el software.
\end{enumerate}

Es importante recalcar que las actividades pueden disminuir o aumentar según se vaya realizando cada una.
\section{Cronograma de ejecución de la obra}

\begin{table}[H]
  \centering
  \caption{Cronograma}
    \begin{tabular}{|l|c|}
    \toprule
    \multicolumn{1}{|c|}{\textbf{Actividades}} & \textbf{Semana} \\
    \midrule
    \multicolumn{1}{|l|}{\multirow{3}[2]{*}{Establecer con el cliente las especificaciones del programa.}} & \multirow{3}[2]{*}{5} \\
          &  \\
          &  \\
    \midrule
    \multirow{2}[2]{*}{Crear el contrato con todos los términos y condiciones.} & \multirow{2}[2]{*}{$6\leftrightarrow7$} \\
          &  \\
    \midrule
    \multirow{2}[2]{*}{Iniciar con el financiamiento del proyecto.} & \multirow{2}[2]{*}{$8\leftrightarrow9$} \\
          &  \\
    \midrule
    \multirow{2}[2]{*}{Crear el software.} & \multirow{2}[2]{*}{$10\leftrightarrow15$} \\
          &  \\
    \midrule
    \multirow{2}[2]{*}{Realizar las pruebas al programa.} & \multirow{2}[2]{*}{$16\leftrightarrow17$} \\
          &  \\
    \midrule
    \multirow{2}[2]{*}{Crear la documentación interna y externa del programa.} & \multirow{2}[2]{*}{$18\leftrightarrow19$} \\
          &  \\
    \midrule
    Entregar el software. & 20 \\
    \bottomrule
    \end{tabular}%
  \label{tab:addlabel}%
\end{table}%


\section{Estudio de factibilidad}

\subsection{Factibilidad Técnica}
 
La empresa desarrolladora cuenta con las herramientas para poder llevar a cabo toda la programación sin necesidad de tener que asumir gastos grandes. El personal de la empresa cuenta con las habilidades y el equipo necesarios para realizar el proyecto esto permite avanzar de forma rápida.
\subsection{Factibilidad Económica}
La desarrolladora no generará grandes gastos ya que tiene las herramientas y el personal para realizar el proyecto. El salario de los empleados será cubierto con un préstamo solicitado a un banco como también de ser necesarios servidores o bases de datos, el mismo se cubrirá con el pago que realicé el cliente después de la entrega final del proyecto. Como también se pondrá a la venta el software para otras empresas esto generará ganancias extras sin la necesidad de solicitar otros préstamos. 
\subsection{Factibilidad Operacional} 
Para desarrollar el proyecto en un tiempo no menor a 3 meses se necesitarán 5 ingenieros en sistemas, esto para que cada uno se centre en uno o varios puntos a realizar del proyecto. Los empleados utilizaran las computadoras de la empresa. Para que el proyecto avance según lo estipulado será necesaria una reunión todos los días para actualizar el avance del mismo. El empleado encargo de los servidores y las bases de datos (de ser requerido) estipulará cuando será necesario realizar la compra de estos.
\subsection{Aprobación de la Solicitud}  
La solicitud del proyecto será aprobada después de comprobar que el préstamo del banco será otorgado y que el cliente firme el contrato de este, esto para asegurar la compra del software.


\section{Requerimientos}

\begin{table}[H]
	\centering
	\caption{Lista de requerimientos funcionales}
\begin{adjustbox}{width=0.45\textwidth}

\begin{tabular}{|c|l|}
\hline 
N° & Nombre Requerimiento \\ 
\hline 
1 & Ordenar \\ 
\hline 
2 & Registro cliente \\ 
\hline 
3 & Aceptar Promociones \\ 
\hline 
4 & Facturación instantánea \\ 
\hline 
5 & Mostrar historial de pedidos anteriores al cliente \\ 
\hline 
6 & Mostrar recomendaciones al cliente según sus pedidos anteriores \\ 
\hline 
7 & Inicio de sesión con usuario y contraseña tanto para clientes como para administradores \\ 
\hline 
8 & Tipo de servicio (Express o restaurante) \\ 
\hline 
9 & Distintos métodos de pago \\ 
\hline 
10 & Capacidad de habilitar la ubicación en tiempo real para mayor precisión en los pedidos express \\ 
\hline

\end{tabular}
\end{adjustbox}
\end{table}%


\begin{table}[H]
	\centering
	\caption{Lista de requerimientos no funcionales}
\begin{adjustbox}{width=0.45\textwidth}

\begin{tabular}{|c|l|}
\hline 
N° & Nombre Requerimiento \\ 
\hline 
1 & Soporte 24/7 \\ 
\hline 
2 & Base de datos \\ 
\hline 
3 & Interfaz sencilla \\ 
\hline 
4 & Mantenimiento a la aplicación \\ 
\hline 
5 & Encriptación de información sensible \\ 
\hline 
6 & Servidor para usuarios, pedidos y facturas \\ 
\hline 
7 & Flexibilidad para hacer cambios al programa \\ 
\hline 
8 & Multiplataforma (Windows, MacOS, Linux) \\ 
\hline 
9 & Capacidad para tener varios usuarios simultáneamente 		tanto para administradores como para clientes \\ 
\hline 
10 & Desarrollo de futuras versiones para sistemas operativos móviles \\ 
\hline

\end{tabular}
\end{adjustbox}
\end{table}%

\subsection{Especificación de requerimientos funcionales}

\begin{enumerate}
%inicio%
\item Nombre del requerimiento:

\begin{enumerate}[a)]
\item Actores
\item Detalle del requerimiento
	\begin{enumerate}[P{a}so 1.]
	\item hola
	\item kola
	\end{enumerate}
\item Reglas del negocio
\end{enumerate}
%fin%

%inicio%
\item Nombre del requerimiento:
\begin{enumerate}[a)]
\item Actores
\item Detalle del requerimiento
	\begin{enumerate}[P{a}so 1.]
	\item hola
	\item kola
	\end{enumerate}
\item Reglas del negocio
\end{enumerate}
%fin%

%inicio%
\item Nombre del requerimiento:
\begin{enumerate}[a)]
\item Actores
\item Detalle del requerimiento
	\begin{enumerate}[P{a}so 1.]
	\item hola
	\item kola
	\end{enumerate}
\item Reglas del negocio
\end{enumerate}
%fin%
\end{enumerate}





\end{document}
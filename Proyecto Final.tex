\documentclass[conference]{IEEEtran}
\usepackage[utf8]{inputenc}
\usepackage{textcomp}
\usepackage[english]{babel}
\usepackage{amsmath}
\usepackage{amsfonts}
\usepackage{amssymb}
\usepackage{graphicx}
%\usepackage{xcolor}
% Dependencies for Excel2Latex
\usepackage[table]{xcolor}
\usepackage{booktabs}
% Dependencies for Excel2Latex
\usepackage{listings}
\usepackage{tikz}
\usepackage{float}
\usepackage{karnaugh-map}
\usepackage{adjustbox}
\usepackage[left=1cm,right=1cm,top=1cm,bottom=1cm]{geometry}
%Habilita bookmarks en PDF
\newcommand\MYhyperrefoptions{bookmarks=true,bookmarksnumbered=true,
pdfpagemode={UseOutlines},plainpages=false,pdfpagelabels=true,
colorlinks=true,linkcolor={black},citecolor={black},
urlcolor={black}}
\usepackage[\MYhyperrefoptions]{hyperref}
%Titulo del documento
\title{Final Project: Using Arduino as a tool to build an automatic pet feeder system}

\makeatletter
\newcommand{\linebreakand}{%
  \end{@IEEEauthorhalign}
  \hfill\mbox{}\par
  \mbox{}\hfill\begin{@IEEEauthorhalign} \hfill
}
\makeatother

\renewcommand\thesection{\arabic{section}}
\renewcommand\thesubsection{\thesection.\arabic{subsection}}
\renewcommand\thesubsubsection{\thesubsection.\arabic{subsubsection}}

\author{
	\IEEEauthorblockN{Angulo Zumbado Jose Daniel}
	\IEEEauthorblockA{\textit{Estudiante Ing. en Sistemas de Computación}\\ 
	\textit{Universidad Fidélitas}\\
	San José, Costa Rica \\
	\href{mailto:jdaz25@gmail.com}{jdaz25@gmail.com}}
\and
	\IEEEauthorblockN{Montoya Abarca Marianela}
	\IEEEauthorblockA{\textit{Estudiante Ing. en Sistemas de Computación}\\ 
	\textit{Universidad Fidélitas}\\
	San José, Costa Rica \\
	\href{mailto:mari.montab@gmail.com}{mari.montab@gmail.com}}
\linebreakand % <------------- \and with a line-break
	\IEEEauthorblockN{Morales Cordero Valeria\hfill}
	\IEEEauthorblockA{\textit{Estudiante Ing. en Sistemas de Computación}\\
	\textit{Universidad Fidélitas}\\
	San José, Costa Rica \\
	\href{mailto:valemc0603@gmail.com}{valemc0603@gmail.com}}
\and
	\IEEEauthorblockN{Raygada Romero Allan Eduardo} 
	\IEEEauthorblockA{\textit{Estudiante Ing. en Sistemas de Computación}\\
	\textit{Universidad Fidélitas}\\
	San José, Costa Rica \\
	\href{mailto:allenraygada@gmail.com}{allenraygada@gmail.com}}

}


%Inicio del documento
\begin{document}

\maketitle

%Agrega numeracion a las paginas
%\thispagestyle{plain}
%\pagestyle{plain}

\begin{abstract}
	
	
\end{abstract}



\section{INTRODUCCIÓN}

\section{METODOLOGÍA}
\subsection{Explicación funcionamiento}


\subsection{Problema que se va a solventar con el proyecto}


\subsection{Posibles riesgos}

\subsection{Lista de componentes del Proyecto:}
\section{DESARROLLO}

\section{Conclusiones}




\section{Recomendaciones}

\section{Referencias}


%%\noindent
%[5] Vidal-Silva, C., Lineros, M. I., Uribe, G. E., \& Olmos, C. J. (2019). Electrónica para Todos con el Uso de Arduino: Experiencias Positivas en la Implementación de Soluciones Hardware- Software. \url{https://doi.org/10.4067/S0718-07642019000600377}\\%%
%
%\noindent
%[6] Mandado, E., Marcos, J., Fernandez, C. and Annesto, J. I., AUTOMATAS Programables Y Sistemas de Automatización, 2009. \url{https://bookdown.org/alberto\_brunete/intro\_automatica/referencias.html}
%
%\noindent
%[6] Tocci, R. (2007). Sistemas digitales Principios y Aplicaciones.\\

\end{document}